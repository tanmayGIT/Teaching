% %%%%%%%%%%%%%%%%%%% PACKAGES  %%%%%%%%%%%%%%%%%%%%
% For characters encoding
%\usepackage[latin1]{inputenc}
\usepackage[T1]{fontenc}
\usepackage{ulem}

% For hyperlinks
%\usepackage{hyperref}
\hypersetup{
    bookmarks=true,         % show bookmarks bar?
    unicode=false,          % non-Latin characters in Acrobat�s bookmarks
    pdfborder={0 0 0},      % border style
    pdftoolbar=true,        % show Acrobat�s toolbar?
    pdfmenubar=true,        % show Acrobat�s menu?
    pdffitwindow=true,     % window fit to page when opened
    pdfstartview={FitV},    % fits the width of the page to the window
    pdfproducer={PDFLaTeX}, % producer of the document
    pdfnewwindow=true,      % links in new window
    % pagebackref=false,     % activate back references inside bibliography. % Must be specified as part of the \usepackage{} statement.
    colorlinks=true,        % false: boxed links; true: colored links
    linkcolor=red,          % color of internal links
    linktoc=all,            % {one,section,page,all} defines which part of an entry in the table of contents is made into a link
    citecolor=green,        % color of links to bibliography
    filecolor=magenta,      % color of file links
    urlcolor=blue,          % color of external links
    % linkbordercolor={1 0 0},% color of frame around internal links (if colorlinks=false)
    % citebordercolor={0 1 0},% color of frame around citations
    % %%%%%%%%%%%%%%%%%%%%%%%%%%%%%%%%%%%%%%%%%%%%%%%%%%%%%%%%%%%%%%%%%%%%%
    % Document related
    % %%%%%%%%%%%%%%%%%%%%%%%%%%%%%%%%%%%%%%%%%%%%%%%%%%%%%%%%%%%%%%%%%%%%%
    pdftitle={Architecture des ordinateurs
    },% title
    pdfauthor={Clement Jonquet
    }, % author
    pdfsubject={Cours IG3 - Polytech' Montpellier - 2014-2015
    }, % subject of the document
    pdfkeywords={
    }, % list of keywords
    pdfcreator={Clement Jonquet
    }, % creator of the document
}

% For graphics
\usepackage{graphicx}

% For alignements in tabular
\usepackage{array}
\usepackage{calc}

% For color in tabular and array
\usepackage{colortbl}

% packages from Ludivine
%\usepackage{ae}
\usepackage{color}
\usepackage[francais]{babel}
\usepackage[footheight=1em]{beamerthemeboxes}
\usepackage{longtable}
\usepackage{ifthen}

% For multirow tables
\usepackage{multirow}

% for digital circuits symbols
%\input milstd

% Conditional compilation
\newif\ifCORR
\ifthenelse{\equal{\detokenize{corr}}{\jobname}}{
\CORRtrue
}{
\CORRfalse
}

\usepackage{hhline}
\usepackage{listings}
\usepackage{mips}
\usepackage{comment}
\usepackage{eurosym}

\definecolor{CommentGreen}{rgb}{0,.6,0}
\lstdefinestyle{mips}
{
  language=[mips]Assembler,
  escapechar=@, % include LaTeX code between `@' characters
  keepspaces,   % needed to preserve spacing with lstinline
  basicstyle=\scriptsize\ttfamily\bfseries,
  commentstyle=\color{CommentGreen},
  stringstyle=\color{cyan},
  showstringspaces=false,
  keywordstyle=[1]\color{blue},    % instructions
  keywordstyle=[2]\color{magenta}, % directives
  keywordstyle=[3]\color{red},     % registers
}

\lstdefinestyle{C}
{
  language=C,
  escapechar=@, % include LaTeX code between `@' characters
  keepspaces,   % needed to preserve spacing with lstinline
  basicstyle=\scriptsize\ttfamily\bfseries,
  commentstyle=\color{CommentGreen},
  stringstyle=\color{cyan},
  showstringspaces=false,
  keywordstyle=[1]\color{blue},    % instructions
  keywordstyle=[2]\color{magenta}, % directives
  keywordstyle=[3]\color{red},     % registers
}

% %%%%%%%%%%%%%%%%%%% COMMANDS %%%%%%%%%%%%%%%%%%%%

\iffalse This is a comment \fi

\newcommand{\non}[1]{\ensuremath{\overline{#1}}}

% %%%%%%%%%%%%%%%%%%% SLIDE PATTERN %%%%%%%%%%%%%%%%%%%%

% 2 columns slide
% \mycolumns{align}{size_c1}{size_c2}{content_c1}{content_c2}
\newcommand{\mycolumns}[5]{
\begin{columns}[#1]
    \column{#2cm}
    #4
    \column{#3cm}
    #5
\end{columns}
}

% 2 columns and 1 row slide
% \mytriangle{figure_size}{column_size}{row_size}{figure}{column}{row}
\newcommand{\mytriangle}[6]{
\begin{center}
\begin{tabular}{m{#2cm}m{#1cm}}
    #5 & \includegraphics[width=#1cm]{figs/#4} \\
    \multicolumn{2}{m{#3cm}}{#6} \\
\end{tabular}
\end{center}
}

% 2 columns and 2 rows slide
% \mygrid{text_size}{image_size}{figure1}{figure2}{text1}{text2}
\newcommand{\mygrid}[6]{
\begin{center}
\begin{tabular}{m{#1cm}m{#2cm}}
    #5 & \includegraphics[width=#2cm]{figs/#3} \\
\end{tabular}
\begin{tabular}{m{#2cm}m{#1cm}}
    \includegraphics[width=#2cm]{figs/#4} & #6\\
\end{tabular}
\end{center}
}

% Command for beginning of sections
% \myatbeginsection{current_or_not}{image_file_name}{image_size}
\newcommand{\myatbeginsection}[3]{
\begin{frame}
    \frametitle{Plan du cours}
    \mycolumns{c}{7}{5}{
        \tableofcontents[#1]
    }{
        \hfill
        \includegraphics[width=#3cm]{figs/#2}
    }
\end{frame}
}

% Command for conceptualmap frame
% \myconceptualmap{image_file_name}
\newcommand{\myconceptualmap}[1]{
\begin{frame}
    \frametitle{De quoi allons-nous \\parler et quel chemin\\ allons nous suivre ?}
    \mycolumns{c}{5}{6.7}{
    }{
        \vspace{-2.1cm}
        \includegraphics[width=67mm]{figs/#1}
     }
\end{frame}
}

% Nearpod note slide
% \nearpodnote{content}
\newcommand{\nearpodnote}[1]{
\note{
    \includegraphics[width=4cm]{figs/nearpod.jpg} \\
    #1
}}

% EClicker note slide
% \eclickernote{content}
\newcommand{\eclickernote}[1]{
\note{
    \includegraphics[width=4cm]{figs/eclicker.png} \\
    #1
}}

% Slide d'exercice r�flexif
% \exoreflexif
\newcommand{\exoreflexif}[0]{
\nearpodnote{
    Exercice r�flexif (3 min)
    \begin{alertblock}{}
        Qu'avez-vous appris de plus important pendant ces chapitres ?
    \end{alertblock}
    \begin{alertblock}{}
        Pourquoi est-ce important selon vous et pour vous ?
    \end{alertblock}
}
}
% Exo r�flexif (3 min) - Qu'avez-vous appris de plus important ? Pourquoi est-ce important pour vous ?

% Slide de proposition de questions
% \propquestion
\newcommand{\propquestion}[0]{
\nearpodnote{
    Proposez vos propres questions pour l'examen.
    \begin{alertblock}{Exercice (5 min)}
        Proposez 4 questions de cours ou petits exercices que vous aimeriez avoir � l'examen ?
    \end{alertblock}
    En lien avec le cours bien sur !
}
}

% Slide d'exercice de recherche de news
% \recherchenews
\newcommand{\recherchenews}[0]{
\nearpodnote{
    \begin{alertblock}{Travail de groupe (5-6 min)}
        Est-ce que quelqu'un parle de notre sujet sur le Web ? \\
        Par groupe de 2-3, recherchez sur le web un article, post ou discussion sur un des sujets que vous pensez en lien a ce cours.\\
        Saisissez son URL et r�sumez le par une courte phrase.
    \end{alertblock}
}
}

%%%%%%%%%%%%%% BEAMER %%%%%%%%%%%%%%%%%%
% Possible blocks : {block} = bleu/violet
%                   {alertblock} = rouge/saumon
%                   {exampleblock} = vert


%%%%%%%%%%%%%% DOCUMENT SETTINGS %%%%%%%%%%%%%%%%%%

\usetheme{Boadilla}
\usefonttheme[onlysmall]{structurebold}
\setbeamercovered{transparent}

% Beamer notes
% To display notes on second screen
\setbeameroption{hide notes}
%\setbeameroption{show notes}
%\setbeameroption{show only notes}
%\setbeameroption{notes on second screen}

% Macros

\def\arduino{\textsf{Arduino}}
\def\clang{\textsf{C}}
\def\claroline{\textsf{Claroline}}
\def\cnam{\textsf{Cnam}}
\def\grove{\textsf{Grove}}
\def\moodle{\textsf{Moodle}}
\def\numalis{\textsf{Numalis}}
\def\pascal{\textsf{Pascal}}
\def\python{\textsf{Python}}
\def\raspberry{\textsf{Raspberry~Pi}}

\setbeamertemplate{navigation symbols}{}
